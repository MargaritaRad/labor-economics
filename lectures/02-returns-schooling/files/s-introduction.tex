\begin{frame}\begin{center}
\LARGE\textbf{Introduction}
\end{center}\end{frame}
%-------------------------------------------------------------------------------
%-------------------------------------------------------------------------------
\begin{frame} I heavily draw on the material presented in:

\begin{itemize}
\item \bibentry{Heckman.2006a}
\end{itemize}

\end{frame}
%-------------------------------------------------------------------------------
%-------------------------------------------------------------------------------
\begin{frame}\textbf{Importance of returns}\vspace{0.3cm}

\begin{itemize}\setlength\itemsep{1em}
\item explain wage inequality within countries
\item explain growth differentials across countries
\item assess schooling investment on individual level
\item evaluate public policies to foster educational attainment
\item ...
\end{itemize}
\end{frame}
%-------------------------------------------------------------------------------
%-------------------------------------------------------------------------------
\begin{frame}\textbf{Core parameter}\vspace{0.3cm}

The internal rate of return is the  discount rate that equates the present value of two potential income streams \cite{Becker.1964}.

\end{frame}
%-------------------------------------------------------------------------------
%-------------------------------------------------------------------------------
\begin{frame}\textbf{Different return concepts}\vspace{0.3cm}

\begin{itemize}\setlength\itemsep{1em}
\item Mincer rate of return
\item internal rate of return
\item true rate of return
\end{itemize}\vspace{0.3cm}

$\Rightarrow$  We will also distinguish between ex ante and ex post returns when introducing uncertainty and the sequential revelation of uncertainty.

\end{frame}
