
 \begin{frame}

 I heavily draw on the material presented in:

\begin{itemize}
\item \bibentry{Heckman.2006a}
\end{itemize}

\end{frame}


 \begin{frame}
We will look at two papers that explore reduced-form estimations of the returns to education
.
\begin{itemize}
\item \bibentry{Carneiro.2002}
\item \bibentry{Mogstad.2017}
\end{itemize}
 \end{frame}

 \begin{frame}
We will look at two papers that explore structural estimations of the returns to education
.
\begin{itemize}
\item \bibentry{Cunha.2005b}
\item \bibentry{Eisenhauer.2015b}
\end{itemize}
 \end{frame}


\begin{frame}
Why are returns to education important?
\begin{itemize}
\item explain wage inequality within countries
\item explain growth differentials across countries
\item assess schooling investment on individual level
\item evaluate public policies to foster educational attainment
\item ...
\end{itemize}
\end{frame}

\begin{frame}\nocite{Mincer.1958,Mincer.1974}
\textbf{Mincer Equation}\\
\begin{align*}
\ln Y(s, x) = \alpha + \rho_s s + \beta_0 x + \beta_1 x^2 + \epsilon\\
\end{align*}
$\Rightarrow$ How to interpret the \textit{Mincer Coefficient} $\rho_s$?
\end{frame}




\begin{frame}
\textbf{Conceptual Frameworks}
\begin{itemize}
\item compensating differences model
\item accounting-identity model
\end{itemize}
\end{frame}

%-------------------------------------------------------------------------------
%-------------------------------------------------------------------------------
\begin{frame}\begin{center}
\LARGE\textit{Compensating Differences Model}
\end{center}\end{frame}

\begin{frame}
\begin{align*}
V(s) = Y(s)\int_s^T e^{-rt} dt = \frac{Y(s)}{r}(e^{-rs} - e^{-rT})
\end{align*}
\end{frame}

\begin{frame}

Equalizing present value of earnings across schooling levels:
\begin{align*}
\ln Y(s) = \ln Y(0) + r s + \ln\left(\frac{1 -e^{-rs}}{1 - e^{-r(T - s)}}\right) \\
\end{align*}

$\Rightarrow$ $\rho_s$ equals the market interest rate and the internal rate of return to schooling by construction.
\end{frame}

\begin{frame}
Model Features:
\begin{itemize}
\item identical abilities and opportunities
\item no credit constraints
\item perfect certainty
\item no direct cost of schooling
\item no nonpecuniary benefits of school and work
\end{itemize}
\end{frame}


\begin{frame}
Model Features:
\begin{itemize}
\item identical abilities and opportunities
\item no credit constraints
\item perfect certainty
\item no direct cost of schooling
\item no nonpecuniary benefits of school and work
\end{itemize}
\end{frame}

%-------------------------------------------------------------------------------
%-------------------------------------------------------------------------------
\begin{frame}\begin{center}
\LARGE\textit{Accounting-Identity Model}
\end{center}\end{frame}

\begin{frame}
\begin{align*}
P_t & \equiv P_{t - 1} (1 + k_{t - 1} \rho_{t - 1}) \equiv \prod^{t - 1}_{j= 0} (1 + \rho_jk_j)P_0 \\
& \\
\ln P_t & \equiv \ln P_0  + s \ln(1 + \rho_s) + \sum^{t -1}_{j=s} \ln(1 + \rho_0 k_j) \\
& \approx  \ln P_0 + s \rho_s + \rho_0 \sum^{t - 1}_{j=s} k_j
\end{align*}
\end{frame}



\begin{frame}
Assuming linearly declining rate of post-school investment:
\begin{align*}
k_{s + x} = \kappa\left( 1 - \frac{x}{T}\right),\text{where} \quad x = t - s
\end{align*}
\end{frame}

\begin{frame}
\begin{figure}[htp]\centering
\caption{Post-School Investment}\label{Post-School Investment}\scalebox{0.30}{\includegraphics{fig-mincer-post-school}}
\end{figure}
\end{frame}

\begin{frame}
 \begin{align*}
 \ln P_{x + s}  \approx \ln P_0 + s\rho_s + \left(\rho_0 \kappa + \frac{\rho_0\kappa}{2T}\right)x - \frac{\rho_0\kappa}{2T} x^2 \\
 \end{align*}
Accounting for the difference in potential and observed earnings:
\begin{align*}
\ln Y(s, x) & = \ln P_{x + s} - \kappa\left(1 - \frac{x}{T}\right) \\
            & = [\ln P_0 - \kappa] + \rho_s s + \left(\rho_0\kappa + \frac{\rho_0\kappa}{2T} + \frac{\kappa}{T}\right) x - \frac{\rho_0\kappa}{2T}x^2
\end{align*}
$\Rightarrow$ $\rho_s$ is the average earnings increase with schooling
\end{frame}

\begin{frame}
\textbf{Standard Mincer Equation}
\begin{align*}
\ln Y(s, x) = \alpha + \rho_s s + \beta_0 x + \beta_1 x^2,
\end{align*}
where
\begin{align*}
\alpha & =\ln P_0 - \kappa \\
\beta_0 & = \left(\rho_0\kappa + \frac{\rho_0\kappa}{2T} + \frac{\kappa}{T}\right) \\
\beta_1 & = -\frac{\rho_0\kappa}{2T}
\end{align*}
\end{frame}

\begin{frame}
\textbf{Random Coefficient Version}
\begin{align*}
\ln Y(s_i, x_i) = \alpha_{i} + \rho_{si} s_i + \beta_{0i} x_i + \beta_{1i} x_i^2
\end{align*}
and let
\begin{align*}
\bar{\alpha} = \E[\alpha_i] &\qquad \bar{\rho}_s = \E[\rho_{si}]\\
\bar{\beta}_0 = \E[\beta_{0i}]&\qquad \bar{\beta}_1 = \E[\beta_{1i}]
\end{align*}
\end{frame}


\begin{frame}
Dropping individual subscripts ...
\begin{align*}
\ln Y(s, x) & = \bar{\alpha} + \bar{\rho}_s s + \bar{\beta}_{0} x + \bar{\beta}_{1} x^2 \\
                & + \underbrace{[(\alpha - \bar{\alpha}) + (\rho_s - \bar{\rho}_s) s + (\beta_0 - \bar{\beta}_0)x + (\beta_1 - \bar{\beta}_1)x^2 ]}_{\epsilon}\\
\end{align*}
$\Rightarrow$ If the schooling decision is determined by individual returns, then we are back in the case of a correlated random coefficient model \cite{Heckman.2006d}.
\end{frame}





\begin{frame}[plain]

\begin{center}
\includegraphics[width=.65\columnwidth]{fig-mincer-regressions}
\end{center}

\end{frame}

%-------------------------------------------------------------------------------
%-------------------------------------------------------------------------------
\begin{frame}\begin{center}
\LARGE\textit{Implications}
\end{center}\end{frame}



\begin{frame}
\begin{itemize}
\item Log-earnings profiles are parallel across schooling levels.
\begin{align*}
\frac{\partial \ln Y(s, x)}{\partial s \partial x} = 0
\end{align*}
\item Log-earnings age profiles diverge with age across schooling levels.
\begin{align*}
\frac{\partial \ln Y(s, x)}{\partial s \partial t} = \frac{\rho_0\kappa}{T} > 0
\end{align*}
\item The variance of earnings over the life cycle has a U-shaped pattern.
\end{itemize}
\end{frame}


\begin{frame}
\begin{figure}[htp]\centering
\caption{Mincerian Experience Profiles}\label{Mincerian Experience Profiles}\scalebox{0.3}{\includegraphics{fig-mincer-equation-experience}}
\end{figure}
\end{frame}

\begin{frame}
\begin{figure}[htp]\centering
\caption{Mincerian Age Profiles}\label{Mincerian Age Profiles}\scalebox{0.3}{\includegraphics{fig-mincer-equation-age}}
\end{figure}
\end{frame}


\begin{frame}
\begin{figure}[htp]\centering
\caption{Mincerian Variance Profiles}\label{Mincerian Variance Profiles}\scalebox{0.3}{\includegraphics{fig-mincer-equation-variance}}
\end{figure}
\end{frame}

%-------------------------------------------------------------------------------
%-------------------------------------------------------------------------------
\begin{frame}\begin{center}
\LARGE\textit{Empirical Evidence}
\end{center}\end{frame}

\begin{frame}[plain]
\begin{center}
\includegraphics[width=.90\columnwidth]{fig-mincer-experience-1940}
\end{center}
\end{frame}

\begin{frame}[plain]
\begin{center}
\includegraphics[width=.90\columnwidth]{fig-mincer-experience-1990}
\end{center}
\end{frame}

\begin{frame}[plain]
\begin{center}
\includegraphics[width=.90\columnwidth]{fig-mincer-parallelism}
\end{center}
\end{frame}

\begin{frame}[plain]
\begin{center}
\includegraphics[width=.90\columnwidth]{fig-mincer-age-1940}
\end{center}
\end{frame}

\begin{frame}[plain]
\begin{center}
\includegraphics[width=.90\columnwidth]{fig-mincer-age-1980}
\end{center}
\end{frame}

\begin{frame}[plain]
\begin{center}
\includegraphics[width=.90\columnwidth]{fig-mincer-variance-1940}
\end{center}
\end{frame}

\begin{frame}[plain]
\begin{center}
\includegraphics[width=.90\columnwidth]{fig-mincer-variance-1980}
\end{center}
\end{frame}


\begin{frame}
In the end, \cite{Heckman.2006a} conclude:\vspace{0.5cm}

\begin{quote}
In common usage, the coefficient on schooling in a regression of log earnings on years of schooling is often called a rate of return. In fact, it is a price of schooling from a hedonic market wage equation. It is a growth rate of market earnings with years of schooling and not an internal rate of return measure, except under stringent conditions which we specify, test and reject in this chapter.
\end{quote}
\end{frame}
%-------------------------------------------------------------------------------
%-------------------------------------------------------------------------------
\begin{frame}\begin{center}
\LARGE\textit{Estimating Internal Rates of Return}
\end{center}\end{frame}
%-------------------------------------------------------------------------------
%-------------------------------------------------------------------------------
\begin{frame}
\textbf{Income Maximization under Perfect Certainty \nocite{Rosen.1977,Willis.1979}}
\begin{align*}
s               &\qquad\text{schooling level} \\
x               &\qquad\text{experience level} \\
Y(s, x)         &\qquad\text{wage income} \\
T(s)            &\qquad\text{last age of earnings} \\
v               &\qquad\text{tuition and psychic cost of schooling} \\
\tau            &\qquad\text{proportional tax rate} \\
r               &\qquad\text{before-tax interest rate}
\end{align*}
\end{frame}

\begin{frame}
\textbf{Present Discounted Value fo Lifetime Earnings}
\begin{align*}
V(s) = & \int_0^{T(s) - s} (1 - \tau) e^{-(1 - \tau)r(x + s)} Y(s,x) dx \\
       & - \int^s_0 ve^{-(1 - \tau)rz}dz
\end{align*}
\end{frame}


\begin{frame}
First-Order Condition
\begin{align*}
& [T^\prime(s) - 1]e^{-(1 - \tau)r(T(s) - s)} Y(s, T(s) - s) \\
& - (1 - \tau)r\int^{T(s) - s}_0 e^{-(1 - \tau)rx} Y(s, x)dx \\
& + \int_0^{T(s) - s} e^{-(1 - \tau) rx} \frac{\partial Y(s, x)}{\partial s}dx \\
& - \frac{v}{ 1  -\tau} = 0
\end{align*}
\end{frame}


\begin{frame}
Rearranging and defining $\tilde{r} = (1 - \tau)r$ ...
\begin{align}
\tilde{r} & = \frac{[T^\prime(s) - 1]e^{-\tilde{r}(T(s) - s)}Y(s, T(s) - s )}{\int_0^{T(s) - s} e^{-\tilde{r}x}Y(s, x) dx} \\
          & + \frac{\int_0^{T(s) - s}e^{-\tilde{r}x}\left[\frac{\partial Y(s, x)}{\partial s}\right] dx}{\int_0^{T(s) - s}e^{-\tilde{r}x}Y(s, x) dx} \\
          & - \frac{\frac{v}{1-\tau}}{\int_0^{T(s) - s}e^{-\tilde{r}x}Y(s, x)dx}
\end{align}
\end{frame}

\begin{frame}
Interpretation
\begin{itemize}
\item (1) ... the change in the present value of earnings due to a change in working-life with additional schooling
\item (2) ... weighted average effect of schooling on log earnings by experience
\item (3) ... tuition and psychic costs expressed as a fraction of lifetime income measured at age $s$
\end{itemize}
All components are expressed as a fraction of the present value of earnings measured at age $s$
\end{frame}


\begin{frame}
Getting back to Mincer ...
\begin{itemize}
\item no tuition and psychic costs of schooling \\
    $\qquad\Rightarrow v = 0$
\item no loss of working life from schooling \\
    $\qquad\Rightarrow T^\prime(s) = 1$
\item multiplicative separability between schooling and experience component of earnings \\
    $\qquad\Rightarrow Y(s, x) = \mu(s)\psi(x)$
\end{itemize}
\end{frame}


\begin{frame}
\begin{align*}
\tilde{r} = \frac{\mu^\prime(s)}{\mu(s)}\quad\forall\quad s\\
\end{align*}

Thus, wage growth must be log linear in schooling and $\mu(s) = \mu(0)e^{\tilde{r}s}$
\end{frame}




\begin{frame}
\cite{Heckman.2006a} thus establish ... \vspace{0.5cm}

\begin{quote}
After allowing for taxes, tuition, variable length of working life, and a flexible relationship between earnings, schooling and experience, the coefficient on years of schooling in a log earnings regression need no longer equal the internal rate of return.
\end{quote}
\end{frame}


\begin{frame}\textbf{Structural Approach for the IRR}\vspace{0.3cm}\\

The internal rate of return for schooling level $s_1$ versus $s_2$, $r(s_1, s_2)$ solves ...

\begin{align*}
&\int_{0}^{T(s_1) - s_1} (1 - \tau)e^{-r(x + s_1)}Y(s_1, x) dx  - \int_{0}^{s_1} v e^{-r z} dz\\
&\qquad\qquad =  \int_{0}^{T(s_2) - s_2} (1 - \tau)e^{-r(x + s_2)}Y(s_2, x) dx - \int_{0}^{s_2} v e^{-r z} dz
\end{align*}

\end{frame}

\begin{frame}
Back to Mincer ....

\begin{itemize}
\item no taxes and no direct or psychic costs of schooling \\\vspace{0.3cm}
\hspace{0.3cm}$\Rightarrow v = 0$ and $\tau = 0$\vspace{0.3cm}
\begin{align*}
&\int_{0}^{T(s_1) - s_1} e^{-r(x + s_1)}Y(s_1, x) dx  = \int_{0}^{T(s_2) - s_2} e^{-r(x + s_2)}Y(s_2, x) dx
\end{align*}
\end{itemize}\end{frame}

\begin{frame}
\begin{itemize}
\item equal work-lives irrespective of years of schooling \\\vspace{0.3cm}
\hspace{0.3cm}$\Rightarrow T = T(s_1) - s_1 = T(s_2) - s_2$\vspace{0.3cm}
\begin{align*}
&\int_{0}^T e^{-r(x + s_1)}Y(s_1, x) dx  = \int_{0}^T e^{-r(x + s_2)}Y(s_2, x) dx
\end{align*}
\end{itemize}
\end{frame}

\begin{frame}
\begin{itemize}
\item parallelism in experience across schooling categories \\\vspace{0.3cm}
\hspace{0.3cm}$\Rightarrow Y(s, x) = \mu(s)\psi(x)$\vspace{0.3cm}
\begin{align*}
&\int_{0}^T e^{-r(x + s_1)} \mu(s)\psi(x) dx  = \int_{0}^T e^{-r(x + s_2)} \mu(s)\psi(x)dx
\end{align*}
\end{itemize}
\end{frame}


\begin{frame}
\begin{itemize}
\item linearity of log earnings in schooling \\\vspace{0.3cm}
\hspace{0.3cm}$\Rightarrow\quad \mu(s) = \mu(0)e^{\rho_s s}$\vspace{0.3cm}
\begin{align*}
&\int_{0}^T e^{-r(x + s_1)} \mu(0)e^{\rho_s s_1}\psi(x) dx  = \int_{0}^T e^{-r(x + s_2)} \mu(0)e^{\rho_s s_2}\psi(x)dx
\end{align*}
\end{itemize}
\end{frame}

\begin{frame}
After some further rearranging ...
\begin{align*}
e^{(\rho_s - r)s_1} & = e^{(\rho_s - r) s_2} \\
\Rightarrow \rho_s  & = r
\end{align*}
\end{frame}

%-------------------------------------------------------------------------------
%-------------------------------------------------------------------------------
\begin{frame}\begin{center}
\LARGE\textit{Empirical Evidence}
\end{center}\end{frame}
%-------------------------------------------------------------------------------
%-------------------------------------------------------------------------------
\begin{frame}[plain]
\begin{center}
\scalebox{0.5}{\includegraphics{tab-irr-specifications-white}}
\end{center}
\end{frame}

\begin{frame}[plain]
\begin{center}
\scalebox{0.5}{\includegraphics{tab-irr-specifications-black}}
\end{center}
\end{frame}

\begin{frame}[plain]
\begin{center}
\scalebox{0.5}{\includegraphics{tab-irr-specifications-cost}}
\end{center}
\end{frame}

\begin{frame}[plain]
\begin{center}
\scalebox{0.75}{\includegraphics{fig-college-tuition}}
\end{center}
\end{frame}

\begin{frame}[plain]
\begin{center}
\scalebox{0.75}{\includegraphics{fig-marginal-taxes}}
\end{center}
\end{frame}

\begin{frame}[plain]
\begin{center}
\scalebox{0.75}{\includegraphics{fig-irr-high-school}}
\end{center}
\end{frame}

\begin{frame}[plain]
\begin{center}
\scalebox{0.75}{\includegraphics{fig-irr-college}}
\end{center}
\end{frame}

\begin{frame}
	\begin{align*}
	\ln{Y(s,x)} & = \ln{P_{s+x}} + \ln{(1 - k_{s+x})} \\
	& \approx \ln{P_s} + \rho_0 \sum^{x - 1}_{j=0} k_{s+j} - k_{s+x}
	\end{align*}
	
	Further, using the assumption of linearly declining investment yields
	
	\begin{align*}
	\ln{Y(s,x)} \approx \ln{P_s} + \kappa \left(\rho_0 \sum^{x - 1}_{j=0} (1 - j/T) - (1 - x/T)\right) \\
	\end{align*}
\end{frame}

\begin{frame}
	Assuming only initial earnings potential $(P_s)$ and investment levels $(\kappa)$ vary in the population, the variance of log earnings is given by
	
	\begin{align*}
	Var(\ln{Y(s,x)}) & = Var(\ln{P_s}) \\
	& + \left(\rho_0 \sum^{x - 1}_{j=0} (1 - j/T) - (1 - x/T)\right)^2 Var(\kappa) \\
	& + 2\left(\rho_0 \sum^{x - 1}_{j=0} (1 - j/T) - (1 - x/T)\right) COV(\ln{P_s},k).
	\end{align*}
\end{frame}

\begin{frame}
	If $\kappa$ and $\ln{P_s}$ are uncorrelated, then earnings are minimized (and equal to $Var(\ln{P_s})$)
	when
	
	\begin{align*}
	\rho_0 \sum^{x - 1}_{j=0} (1 - j/T) = 1 - x/T, or \\
	\\
	\rho_0\left(x - \frac{x(x - 1)}{2T}\right) = \rho_0(1 - x/T).
	\end{align*}
\end{frame}

\begin{frame}
	Clearly, $\lim_{T\to\infty} x^* = \frac{1}{\rho_0}$, so the variance minimizing age is $\frac{1}{\rho_0}$ when the work-life is long. More generally, re-arranging terms and solving for the root of this equation, yields the variance minimizing experience level of \\
	
	\begin{align*}
	x^* & = T + \frac{1}{2} + \frac{1}{\rho_0} - \sqrt{\left(T + \frac{1}{2} + \frac{1}{\rho_0}\right)^2 - \frac{2T}{\rho_0}} \\
	& \approx \left(\rho_0 + \frac{\rho_0}{2T} + \frac{1}{T}\right)^{-1},
	\end{align*}
\end{frame}

\begin{frame}
	The final approximation comes from a first-order Taylor approximation of the square root term around the squared term inside. The approximation suggests that the variance minimizing age will generally be less than or equal to $\frac{1}{\rho_0}$, with the difference disappearing as $T$ grows large.
\end{frame}
